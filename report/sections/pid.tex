%!TEX root = ../report.tex

\section{PID}

\subsection{Création et évolution au long du temps}

Le régulateur PID est un système de contrôle. Il porte ce nom à cause de ses trois actions (action proportionnelle, intégrale et dérivée). 

Les premières analyses techniques de PID datent de 1922, lors que Nicolas Minorsky essaye de créer des systèmes de pilotage automatique pour la marine des États-Unis. \cite{minorsky1922directional}

Depuis son invention, la popularité de ce régulateur a fortement augmenté. \cite{ang2005pid} PID et ses variantes sont les systèmes de contrôle les plus utilisés (95\% des régulateurs dans les procédés industriels). \cite{Kinnaert2013} \cite{Astrom2002}

Malgré le fait que de nombreuses autres techniques ont été inventées depuis la création de PID, il maintient sa structure de base et a simplement évolué avec le temps: il existe à présent soit en tant que bloc fonctionnel dans des contrôleurs logiques programmables, soit en tant que contrôleur ``standalone'' (et non pas comme composante pneumatique ou électrique). \cite{visioli2006practical}

PID maintient sa popularité car il est considéré qu'aucun autre contrôleur n'égale la simplicité, fonctionnement clair, facilité d'application et utilisation offerte par PID. Il sert aussi de composante fondamentale à d'autres contrôleurs plus complexes. \cite{ang2005pid} \cite{visioli2006practical}

\subsection{Rôle de PID}

Le rôle de PID est de calculer en continu une erreur. Cette erreur est la différence entre une variable mesurée et un point désiré. PID tente de minimiser l'erreur au long du temps en s'ajustant peu à peu. Étant donné que PID se base uniquement sur des variables mesurées, sans connaissance du procédé sous-jacent, il a énormément d'applications différentes. \cite{bennett1993history}

Le grand avantage de PID est que l'on peut ajuster certains de ses paramètres pour s'adapter à la réalité du projet. Les effets (positifs ou négatifs) de l'ajustement de ces paramètres sont, entre autres: \cite{zhong2006pid}
\begin{enumerate}
\item la stabilité,
\item précision (erreur d'état d'équilibre),
\item rapidité (lors du lancement ou du d'arrivée à la stabilité)
\end{enumerate}

Il est aussi possible, dans quelques applications de PID, que certaines actions soient inutiles. Dans ce cas-là, on parle de contrôleurs PI, PD, P ou I. Par exemple, une des raisons d'enlever l'action dérivée est qu'elle est fort sensible au bruit lors des mesures. (voir ``Alternatives à PID'' ).

\newpage