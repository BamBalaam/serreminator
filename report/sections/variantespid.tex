%!TEX root = ../report.tex

\section{Variantes de PID}

\subsection{Integral windup}
Le problème de l'Intgral Windup survient lorsqu'un algorithme de contrôle mal programmé demande une correction impossible, tel que l'ouverture d'une vanne au-delà de son maximum, ou encore la diminution de rythme d'un ventilateur déjà éteint. La partie \emph{Intégration} de PID sommant les corrections, ces erreurs vont s'accumuler et provoquer un \emph{overshoot}, c'est à dire une correction au-delà des limites possibles.

Ce phénomène étudié depuis des années fut résolu par de nombreux moyens, comme le démontrent \cite{ControlGuruIntegralWindup}, \cite{astrom1995pid}, \cite{shin2012anti}, ou encore \cite{bohn1995analysis}.

On peut citer quelques exemples de corrections, telles que:
\begin{description}
\item[limiter les bornes de variations] une façon simple de résoudre ce problème est de ne pas envoyer de commandes en dehors de bornes minimums et maximums; cela conduit néanmoins à des limitations de performances, ces bornes ne réglant pas suffisamment rapidement la perturbation
\item[algorithme incrémental] la correction est d'abord calculée par un algorithme, et ensuite suivant le résultat, sera envoyée dans l'intégrateur; le windup est alors corrigé en enlevant l'action de l'intégrateur tant que la sortie est trop haute/basse
\item[back-calculation] lorsque la sortie est en dehors des bornes, celle-ci est limitée aux bornes et sauvegardée pour être considérée dans le prochain tour de la boucle de contrôle
\end{description}

\subsection{Derivative kick}

\subsection{Gain scheduling}

\subsection{fuzzy logic}

\section{Alternatives à PID}

\subsection{Bang bang control}
le \emph{Bang bang control}, aussi appelé \emph{On-Off control} ou en français, \emph{Tout ou rien}, est un controleur qui ne peut accepter que deux états de modification, tels que ouvert ou fermé, ou allumé ou éteint.

Des exemples très classiques d'utilisation de ce type de contrôleur sont les thermostats: le chauffage s'allume sous une température minimale, et s'éteint au-dessus d'un seuil maximal.

\subsection{P controller}
Le \emph{Proportional Pontrol} est l'ancêtre du PID, il ne prend en compte que la partie proportionnelle de celui-ci. Cet algorithme se situe entre l'algorithme Bang Bang et PID: en effet, là où Bang Bang va corriger l'état simplement en allumant ou en éteignant un appareil, l'algorithme P control va lui, appliquer une réponse appropriée à la perturbation.

Un tel contrôleur s'exprime par l'équation $P_{out} = K_{p}e(t) + p0$ où $e(t) = SP - SV$ est l'erreur mesurée (différence entre valeur attendue et valeur reçue), $K_{p}$ est le paramètre de gain proportionnel, $P_{out}$ est la réponse à la perturbation et $p0$ est la correction à appliquer, nécessaire vu que cet algorithme n'a pas de composante intégrale par rapport à PID.

Un grand avantage à l'utilisation de ce type de contrôleur est qu'il n'y a qu'un seul paramètre à configurer, qui définit à quel point la correction sera agressive: plus le paramètre $K_{p}$ est petit, plus la réaction est lente, et à l'inverse, plus il est grand, plus ce sera rapide.

Si l'implémentation d'un tel contrôleur est aisée, ses sorties produisent un phénomène d'\emph{offset}, un décalage par rapport aux valeurs attendues.

Le choix d'un tel contrôleur par rapport à PID dépend de l'utilisation, de la précision et de la vitesse des corrections désirées. Des exemples où l'algorithme P control est suffisant sont donnés dans \cite{sellers2001overview}.

\subsection{Integral controller}
Le principe d'un \emph{Integral controller} est de corriger un offset résultant de l'utilisation d'un P controller. Un tel contrôleur est caractérisé dans \cite{svrcek2006real} par l'équation $\frac{1}{T_{i}}\int e dt + MV_{0}$ où $MV_{0}$ correspond à la correspond biaisée de P controller, $\int e dt$ représente l'intégrale des erreurs sur l'intervalle de temps $dt$, et $T_{i}$ est le temps intégral, défini comme le temps nécessaire pour changer la sortie du contrôleur d'une quantité égale à l'erreur.

\subsection{PI controller}
Le contrôleur PI est un cas particulier de PID dans lequel la partie dérivée n'est pas prise en compte. 
Un tel système sera moins réactif aux perturbations, que ce soit du point de vue réponse ou temps de correction.
