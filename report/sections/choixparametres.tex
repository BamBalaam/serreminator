%!TEX root = ../report.tex
\section{Choix de paramètres dans PID}
PID nécessite 3 paramètres:
\begin{itemize}
	\item $K_p$ aussi appelé P. Il montre l'importance par rapport aux valeurs présentes.
	\item $K_i$ aussi appelé I. Il fait varier la réponse de PID en fonction des erreurs passées.
	\item $K_d$ aussi appelé D. Il prend en compte les valeurs futures attendues par rapport aux changement actuellement.
\end{itemize}
Pour déterminer leurs valeurs optimales, plusieurs algorithmes peuvent être utilisés.

\subsection{La méthode manuelle}
Cette technique s'applique sur un système en live.
L'avantage de cette technique, c'est quelle ne nécessite aucun calcul mathématique cependant, il faut des personnes expérimentées pour s'en occuper.
L'algorithme se passe en 4 étapes:
\begin{enumerate}
	\item D'abord, on met les valeurs des 3 paramètres a 0.
	\item Puis, on augmente le paramètre $K_p$ jusqu'à ce que la sortie de la boucle oscille.
	On prendra comme valeur pour $K_p$ la moitié de celle obtenue précédemment.
	\item Ensuite, on augmente $K_i$ jusqu'au moment ou l'offset est corrigé dans un temps suffisant pour le système.
	\item Pour finir, on augmente, si nécessaire, $K_d$ jusqu'au moment où la boucle est suffisamment rapide pour atteindre sa valeur de référence après une perturbation.
\end{enumerate}

\subsection{Ziegler–Nichols}
% http://www.chem.mtu.edu/~tbco/cm416/zn.html
Cette technique commence comme la méthode manuelle:
\begin{enumerate}
	\item On commence par mettre les paramètres à 0.
	\item On augmente $K_p$ jusqu'à ce que la sortie de la boucle oscille (comme précédemment).
\end{enumerate}
Par contre, maintenant on va nommer cette valeur ainsi obtenue $K_u$ et la période de l'oscillation $P_u$.

Ensuite, il suffit de calculer les autres paramètres grâce au tableau \ref{tab:ZieglerNicholsTuningFormulas}

\def\tabularxcolumn#1{m{#1}}
\begin{figure}[ht]
	\begin{center}
		\begin{tabularx}{\textwidth}{| c | X | X | X |}
			\hline
			& $K_p$ & $K_i$ & $K_d$\\ \hline
			P & \begin{equation*}\frac{K_u}{2}\end{equation*} & &\\ \hline
			PI & \begin{equation*}\frac{K_u}{2,2}\end{equation*} & \begin{equation*}1,2 \cdot \frac{K_p}{P_u}\end{equation*} &\\ \hline
			PID & \begin{equation*}\frac{K_u}{1,7}\end{equation*} & \begin{equation*}2 \cdot \frac{K_p}{P_u}\end{equation*} & \begin{equation*}K_p \cdot \frac{P_u}{8}\end{equation*} \\
			\hline
		\end{tabularx}
	\end{center}
	\caption{Tableau des formules pour Ziegler–Nichols}
	\label{tab:ZieglerNicholsTuningFormulas}
\end{figure}

\subsection{Nyquist}

\subsection{La méthode de Tyreus Luyben}
