%!TEX root = ../report.tex
\section{Choix de paramètres dans PID}

Pour déterminer les valeurs optimales des paramètres de PID, plusieurs algorithmes peuvent être utilisés. Nous en présenterons 3 : la méthode manuelle, la méthode de Ziegler–Nichols et celle de Tyreus Luyben.

\subsection{La méthode manuelle}
Cette technique s'applique sur un système en fonctionnement.
L'avantage de cette technique, c'est qu'elle ne nécessite aucun calcul mathématique complexe.
Cependant, elle nécessite l'interaction d'un technicien expérimenté.

L'algorithme se déroule en 4 étapes:
\begin{enumerate}
	\item Les valeurs des 3 paramètres sont fixées à $0$.
	\item Le paramètre $K_p$ est incrémenté jusqu'à ce que la sortie du système se mette à osciller.
	On prendra comme valeur pour $K_p$ la moitié de celle obtenue précédemment.
	\item $K_i$ est augmenté jusqu'au moment où l'offset est corrigé dans un temps acceptable pour le système.
	\item Si nécessaire, $K_d$ est augmenté jusqu'au moment où la boucle est suffisamment rapide pour atteindre à nouveau la consigne après une perturbation extérieure.
\end{enumerate}

\subsection{Ziegler–Nichols}
% ziegler1942optimum
% silva2007pid
% http://www.chem.mtu.edu/~tbco/cm416/tuning_methods.pdf
Comme la méthode manuelle, Ziegler–Nichols s'applique aussi sur un système en fonctionnement. Cependant, elle utilise des calculs un peu plus complexes que la méthode manuelle mais son avantage est qu'elle nécessite des techniciens moins expérimentés.\cite{ziegler1942optimum}\cite{silva2007pid}

Elle débute comme la méthode manuelle :
\begin{enumerate}
	\item Les valeurs des 3 paramètres sont fixées à $0$.
	\item $K_p$ est augmenté jusqu'à ce que la sortie de la boucle oscille (comme précédemment).
\end{enumerate}

Par contre, cette fois-ci, on va nommer la valeur de $K_p$ ainsi obtenue $K_u$ et la période de l'oscillation $P_u$.
Ensuite, les autres paramètres sont déterminés à l'aide du tableau \ref{tab:ZieglerNicholsTuningFormulas}

\def\tabularxcolumn#1{m{#1}}
\begin{figure}[ht]
	\begin{center}
		\begin{tabularx}{\textwidth}{| c | X | X | X |}
			\hline
			& $K_p$ & $K_i$ & $K_d$\\ \hline
			P & \begin{equation*}\frac{K_u}{2}\end{equation*} & &\\ \hline
			PI & \begin{equation*}\frac{K_u}{2,2}\end{equation*} & \begin{equation*}1,2 \cdot \frac{K_p}{P_u}\end{equation*} &\\ \hline
			PID & \begin{equation*}\frac{K_u}{1,7}\end{equation*} & \begin{equation*}2 \cdot \frac{K_p}{P_u}\end{equation*} & \begin{equation*}K_p \cdot \frac{P_u}{8}\end{equation*} \\
			\hline
		\end{tabularx}
	\end{center}
	\caption{Tableau des formules pour Ziegler–Nichols}
	\label{tab:ZieglerNicholsTuningFormulas}
\end{figure}

\subsection{Tyreus Luyben}
%http://www.chem.mtu.edu/~tbco/cm416/tuning_methods.pdf
Cette méthode est fort similaire a celle de Ziegler-Nichols. La plus grande différence est la formule utilisée.\cite{shahrokhi2013comparison}

\def\tabularxcolumn#1{m{#1}}
\begin{figure}[ht]
	\begin{center}
		\begin{tabularx}{\textwidth}{| c | X | X | X |}
			\hline
			& $K_p$ & $K_i$ & $K_d$\\ \hline
			PI & \begin{equation*}\frac{K_u}{3,2}\end{equation*} & \begin{equation*}\frac{K_p}{2,2 \cdot P_u}\end{equation*} &\\ \hline
			PID & \begin{equation*}\frac{K_u}{2,2}\end{equation*} & \begin{equation*}\frac{K_p}{2,2 \cdot P_u}\end{equation*} & \begin{equation*}K_p \cdot \frac{P_u}{6,3}\end{equation*} \\
			\hline
		\end{tabularx}
	\end{center}
	\caption{Tableau des formules pour Ziegler–Nichols}
	\label{tab:ZieglerNicholsTuningFormulas}
\end{figure}
