%!TEX root = ../report.tex
\section{Choix de paramètres dans PID}

Pour déterminer les valeurs optimales des paramètres de PID, plusieurs algorithmes peuvent être utilisés.

\subsection{La méthode manuelle}
Cette technique s'applique sur un système en live.
L'avantage de cette technique, c'est quelle ne nécessite aucun calcul mathématique complexe.
Cependant, elle nécessite l'interaction d'un technicien expérimenté.

L'algorithme se déroule en 4 étapes:
\begin{enumerate}
	\item Les valeurs des 3 paramètres sont fixées à $0$.
	\item Le paramètre $K_p$ est incrémenté jusqu'à ce que la sortie du système se mette à osciller.
	On prendra comme valeur pour $K_p$ la moitié de celle obtenue précédemment.
	\item $K_i$ est augmenté jusqu'au moment ou l'offset est corrigé dans un temps acceptable pour le système.
	\item Si nécessaire, $K_d$ est augmenté jusqu'au moment où la boucle est suffisamment rapide pour atteindre la consigne après une perturbation extérieure.
\end{enumerate}

\subsection{Ziegler–Nichols}
% http://www.chem.mtu.edu/~tbco/cm416/zn.html
La technique de Ziegler–Nichols commence comme la méthode manuelle :
\begin{enumerate}
	\item Les valeurs des 3 paramètres sont fixées à $0$.
	\item $K_p$ est augmenté jusqu'à ce que la sortie de la boucle oscille (comme précédemment).
\end{enumerate}

Par contre, cette fois ci, on va nommer cette valeur de $K_p$ ainsi obtenue $K_u$ et la période de l'oscillation $P_u$.
Ensuite, les autres paramètres sont déterminés à l'aide du tableau \ref{tab:ZieglerNicholsTuningFormulas}

\def\tabularxcolumn#1{m{#1}}
\begin{figure}[ht]
	\begin{center}
		\begin{tabularx}{\textwidth}{| c | X | X | X |}
			\hline
			& $K_p$ & $K_i$ & $K_d$\\ \hline
			P & \begin{equation*}\frac{K_u}{2}\end{equation*} & &\\ \hline
			PI & \begin{equation*}\frac{K_u}{2,2}\end{equation*} & \begin{equation*}1,2 \cdot \frac{K_p}{P_u}\end{equation*} &\\ \hline
			PID & \begin{equation*}\frac{K_u}{1,7}\end{equation*} & \begin{equation*}2 \cdot \frac{K_p}{P_u}\end{equation*} & \begin{equation*}K_p \cdot \frac{P_u}{8}\end{equation*} \\
			\hline
		\end{tabularx}
	\end{center}
	\caption{Tableau des formules pour Ziegler–Nichols}
	\label{tab:ZieglerNicholsTuningFormulas}
\end{figure}

\subsection{Nyquist}

\subsection{La méthode de Tyreus Luyben}
