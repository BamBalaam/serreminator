%!TEX root = ../report.tex
\section{Choix de paramètres dans PID}
Pour fonctionner parfaitement, l'algorithme PID nécessite 3 paramètres:
\begin{itemize}
\item $K_p$ aussi appelé P. Il montre l'importance par rapport aux valeurs présentes.
\item $K_i$ aussi appelé I. Il fait varier la réponse de PID en fonction des erreurs passées.
\item $K_d$ aussi appelé D. Il prend en compte les valeurs futures attendues par rapport aux changement actuellement.
\end{itemize}
 Pour se faire, plusieurs algorithmes peuvent être utilisés.
 \subsection{La méthode manuelle}
 
 \subsection{La méthode de Ziegler–Nichols}
 
 \subsection{La méthode de Ziegler–Nichols inversée}

 \subsection{Nyquist} 

 \subsection{La méthode de Tyreus Luyben}
 
 