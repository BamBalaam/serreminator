\documentclass[a4paper,10pt]{report}
\usepackage[french]{babel}
\usepackage[utf8]{inputenc}
\usepackage[left=2.5cm,top=2cm,right=2.5cm,nohead,nofoot]{geometry}
\usepackage{url}
\usepackage[T1]{fontenc}
\usepackage{float}
\usepackage{afterpage}
\usepackage{amsmath}
\usepackage{graphicx}
\usepackage{tabularx}
\usepackage{csquotes}

\linespread{1.1}

\setlength{\parskip}{0.5em}

\begin{document}

\begin{titlepage}
\begin{center}
\textbf{\textsc{UNIVERSITÉ LIBRE DE BRUXELLES}}\\
\textbf{\textsc{Faculté des Sciences}}\\
\textbf{\textsc{Département d'Informatique}}
\vfill{}\vfill{}
\begin{center}{\Huge Contrôle automatique d'une serre}\end{center}{\Huge \par}
\begin{center}{\large A. Caccia, A. Madeira Cortes, N. Marchant, R. Fontaine}\end{center}{\Huge \par}
\vfill{}\vfill{}
\begin{flushleft}{\large \textbf{Superviseurs :} M. Labbé, T. Lenaerts}\hfill{}\end{flushleft}{\large\par}
\vfill{}\vfill{}\enlargethispage{3cm}
\textbf{Année académique 2015--2016}
\end{center}
\end{titlepage}

\begin{abstract}
Ce rapport présente ...
\end{abstract}


\tableofcontents


\chapter{Introduction}
1.Une introduction contexte + la description du méthodologie étudié,

\chapter{Etat de l'art}
2.l’état de l’art,

\chapter{Méthodes implémentées}
3.un rapport sur les méthodes investiguées et leurs fonctionnement

\chapter{Résultats expérimentaux}
4.les résultats expérimentaux, + un comparaison avec des approches existant.

\chapter{Discussion}
5.l’explication/discussion de ces résultats

\chapter{Conclusion et perspectives}
6.conclusions et perspectives.

\bibliographystyle{apalike}

\bibliography{mybiblio}
\addcontentsline{toc}{chapter}{Bibliographie}

\end{document}
