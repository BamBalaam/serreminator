\documentclass[a4paper,10pt]{report}
\usepackage[french]{babel}
\usepackage[utf8]{inputenc}
\usepackage[left=2.5cm,top=2cm,right=2.5cm,nohead,nofoot]{geometry}
\usepackage{url}
\usepackage[T1]{fontenc}
\usepackage{float}
\usepackage{afterpage}
\usepackage{amsmath}
\usepackage{graphicx}
\usepackage{tabularx}
\usepackage{csquotes}
\usepackage{fullpage}

\usepackage[pdftex,
            pdfauthor={A. Caccia, A. Madeira Cortes, N. Marchant, R. Fontaine},
            pdftitle={Contrôle automatique d'une serre},
            pdfsubject={Contrôle automatique d'une serre}]{hyperref}

\linespread{1.1}

\setlength{\parskip}{0.5em}

\begin{document}

\begin{titlepage}
    \begin{center}
        \textbf{\textsc{Université Libre de Bruxelles}}\\
        \textbf{\textsc{Faculté des Sciences}}\\
        \textbf{\textsc{Département d'Informatique}}

        \vfill{}
        \vfill{}

        \begin{center}
            {\Huge Contrôle automatique d'une serre}
        \end{center}

        {\Huge \par}

        \begin{center}
            {\large A. Caccia, A. Madeira Cortes, N. Marchant, R. Fontaine}
        \end{center}

        {\Huge \par}
        \vfill{}
        \vfill{}

        \begin{flushleft}
            {\large \textbf{Superviseurs :} M. Labbé, T. Lenaerts}
            \hfill{}
        \end{flushleft}

        {\large\par}
        \vfill{}
        \vfill{}
        % \enlargethispage{3cm} % do not remove

        \textbf{Année académique 2015--2016}
    \end{center}
\end{titlepage}

\begin{abstract}
Ce rapport présente ...
\end{abstract}


\tableofcontents


\chapter{Introduction}
1.Une introduction contexte + la description du méthodologie étudié,

2 pages

\chapter{Etat de l'art}
2.l’état de l’art,

\section{Contrôle automatique}

\section{Détermination des paramètres de PID}

\chapter{Méthodes implémentées}
3.un rapport sur les méthodes investiguées et leurs fonctionnement

\section{Contrôle automatique}
BangBang
PID

\section{Détermination des paramètres de PID}

Ziegler + alternatives
Genetic
Tabu
Manuel

\chapter{Résultats expérimentaux}
4.les résultats expérimentaux, + un comparaison avec des approches existant.

\section{Méthode expérimentale}
Déterminer des métriques
    Temps de montée
    ampiltude max
    integrale de l'erreur

\section{Résultats}
Résulatas pour chaque méthode

pour genetic : comparer en fct du nombre de générations (et le de taille de la pop)

\chapter{Discussion}
5.l’explication/discussion de ces résultats

\chapter{Conclusion et perspectives}
6.conclusions et perspectives.

\bibliographystyle{apalike}

\bibliography{mybiblio}
\addcontentsline{toc}{chapter}{Bibliographie}

\end{document}
