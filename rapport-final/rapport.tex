\documentclass[a4paper,10pt]{report}
\usepackage[french]{babel}
\usepackage[utf8]{inputenc}
\usepackage[left=2.5cm,top=2cm,right=2.5cm,nohead,nofoot]{geometry}
\usepackage{url}
\usepackage[T1]{fontenc}
\usepackage{float}
\usepackage{afterpage}
\usepackage{amsmath}
\usepackage{graphicx}
\usepackage{tabularx}
\usepackage{csquotes}
\usepackage{fullpage}

\usepackage[pdftex,
            pdfauthor={A. Caccia, A. Madeira Cortes, N. Marchant, R. Fontaine},
            pdftitle={Contrôle automatique d'une serre},
            pdfsubject={Contrôle automatique d'une serre}]{hyperref}

\linespread{1.1}

\setlength{\parskip}{0.5em}

\begin{document}

\begin{titlepage}
    \begin{center}
        \textbf{\textsc{Université Libre de Bruxelles}}\\
        \textbf{\textsc{Faculté des Sciences}}\\
        \textbf{\textsc{Département d'Informatique}}

        \vfill{}
        \vfill{}

        \begin{center}
            {\Huge Contrôle automatique d'une serre}
        \end{center}

        {\Huge \par}

        \begin{center}
            {\large A. Caccia, A. Madeira Cortes, N. Marchant, R. Fontaine}
        \end{center}

        {\Huge \par}
        \vfill{}
        \vfill{}

        \begin{flushleft}
            {\large \textbf{Superviseurs :} M. Labbé, T. Lenaerts}
            \hfill{}
        \end{flushleft}

        {\large\par}
        \vfill{}
        \vfill{}
        % \enlargethispage{3cm} % do not remove

        \textbf{Année académique 2015--2016}
    \end{center}
\end{titlepage}

\begin{abstract}
Ce rapport présente ...
\end{abstract}


\tableofcontents


\chapter{Introduction}
1.Une introduction contexte + la description du méthodologie étudié,

2 pages

\chapter{Etat de l'art}

Cet état de l'art est séparé en deux parties : la première détaille les différentes méthodes de régulation automatique et la seconde étudie les différentes méthodes pour configurer l'algorithme \emph{PID}, détaillé en \ref{PID}.

\section{Contrôle automatique}
L’automatique est une science qui étudie la modélisation, l’analyse et la commande de systèmes dynamiques.
L’automatique permet de contrôler un système en respectant un cahier des charges (rapidité, précision, stabilité, ...).

L'automatique était un sujet assez vaste, nous nous limiterons ici à l'étude des différentes méthodes de contrôle permettant de stabiliser l'état d'un système au plus proche d'une valeur consigne.

Nous étudierons premièrement \emph{BangBang}, l'algorithme le plus simpliste, utilisé dans les thermostats.
Ensuite nous étudierons \emph{PID} ainsi que ses variantes.
Pour finir nous survolerons \emph{MPC}, un algorithme prédicitf plus avancé.

\subsection{\emph{Bang bang}}
Le \emph{``Bang bang control''} aussi appelé \emph{``On-Off control''} ou en français \emph{``Tout ou rien''} est un contrôleur qui ne peut accepter que deux états de contrôle tels que ouvert ou fermé, allumé ou éteint.

Des exemples très classiques d'utilisation de ce type de contrôleur sont les thermostats: le chauffage s'allume sous une température minimale, et s'éteint au-dessus d'un seuil maximal.

Nonobstant la facilité d'implémentation et de mise en place de cet algorithme, son utilisation ne peut être généralisée au contrôle de phénomènes spontanés. En effet, si l'on prend l'exemple du contrôle de la luminosité, \emph{bang bang} ne pourra qu'allumer ou éteindre des lampes, il ne pourra influer sur leur intensité. \cite{Burghes2004}

De plus, cet algorithme ne permet pas une stabilité suffisante pour certains phénomènes: en effet, une fois la correction atteinte, \emph{Bang bang} agira plus jusqu'à ce que les pertubations soient trop importantes. \cite{Burghes2004} % TODO : double citation en 2§


\subsection{\emph{PID}}
\label{PID}

\emph{PID} est le régulateur le plus connu et le plus utilisé dans le domaine industriel : il serait utilisé dans plus de 95\% des systèmes \cite{Kinnaert2013, aastrom2002control}.
Il porte ce nom à cause de son fonctionnement qui est découpé en trois actions : l'action proportionnelle, l'action intégrale et l'action dérivée.

% TODO : ajouter formule

\subsubsection{Création et évolution au long du temps}

Les premières analyses techniques de \emph{PID} datent de 1922, lorsque Nicolas Minorsky essaye de créer des systèmes de pilotage automatique pour la marine des États-Unis. \cite{minorsky1922directional}

Malgré le fait que de nombreuses autres techniques ont été inventées depuis la création de \emph{PID}, il maintient son fonctionnement de base et a simplement évolué avec le temps :
il existe à présent soit en tant que bloc fonctionnel dans des contrôleurs logiques programmables, soit en tant que contrôleur ``standalone''.

\emph{PID} maintient sa popularité car il est souvent admis qu'aucun autre contrôleur n'égale la simplicité, le fonctionnement clair, la facilité d'application et d'utilisation offerte par \emph{PID}. % TODO source

\emph{PID} sert aussi de composant de base à d'autres contrôleurs plus complexes \cite{ang2005pid} \cite{visioli2006practical}. % TODO trop court


Étant donné que \emph{PID} se base uniquement sur des variables mesurées, sans connaissance du procédé sous-jacent, il a énormément d'applications différentes \cite{bennett1993history}.

Le grand avantage de \emph{PID} est que l'on peut ajuster certains de ses paramètres pour s'adapter à la réalité du projet. % TODO source

Il est aussi possible dans quelques applications de PID que certaines actions soient inutiles.
Dans ce cas-là, on parle de contrôleurs PI, PD, P ou I.
Par exemple, une des raisons d'enlever l'action dérivée est qu'elle est fort sensible au bruit lors des mesures. % TODO source

\subsection{P controller}
Le \emph{Proportional Control} est l'ancêtre du PID, il ne prend en compte que la partie proportionnelle de celui-ci.
Cet algorithme se situe entre l'algorithme Bang Bang et PID:
en effet, là où \emph{Bang Bang} va corriger l'état simplement en allumant ou en éteignant un appareil, l'algorithme P control va lui, appliquer une réponse appropriée à la perturbation.

Un tel contrôleur s'exprime par l'équation
\begin{equation}P_{out} = K_{P}e(t) + p_0\end{equation}
dans laquelle $e(t)$ est l'erreur (la différence entre valeur attendue et valeur mesurée), $K_{P}$ est le paramètre de gain proportionnel, $P_{out}$ est la réponse à la perturbation et $p_0$ est la correction à appliquer, nécessaire vu que cet algorithme n'a pas de composante intégrale (contrairement à PID).

Un grand avantage de l'utilisation de ce type de contrôleur est qu'il n'y a qu'un seul paramètre à configurer, qui définit à quel point la correction sera agressive : plus le paramètre $K_{p}$ est petit (respectivement grand), plus la réaction est lente (respectivement rapide). % TODO source

Si l'implémentation d'un tel contrôleur est aisée, ses sorties produisent un phénomène d'\emph{offset} : un décalage par rapport aux valeurs attendues.

Le choix d'un tel contrôleur par rapport à PID dépend de l'utilisation, de la précision et de la vitesse des corrections désirées.
Des exemples où l'algorithme P control est suffisant sont donnés dans \cite{sellers2001overview}. % TODO : montrer que la souce est utilisée partout dans la subsection

\subsection{\emph{Integral controller}}
Le principe d'un \emph{Integral controller} est de corriger un offset résultant de l'utilisation d'un P controller.
Un tel contrôleur est caractérisé dans \cite{svrcek2014real} par l'équation
\begin{equation}P_{out} = \frac{1}{T_{i}}\int e dt + MV_{0}\end{equation}
où $MV_{0}$ correspond à la correction biaisée de P controller,
$\int e dt$ représente l'intégrale des erreurs sur l'intervalle de temps $dt$ et $T_{i}$ est le temps intégral défini comme le temps nécessaire pour changer la sortie du contrôleur d'une quantité égale à l'erreur.

Bien que ce contrôleur propose une correction aux décalages, on observe un temps de réponse jusqu'à dix inférieur inférieur à l'utilisation d'un P controller seul \cite{svrcek2014real}.

\subsection{PI controller}
Un contrôleur PI utilise à la fois l'action proportionnelle et l'action intégrale.
Il est caractérisé par l'équation
\begin{equation}P_{out} = K_{P} e + K_{I} \int e dt\end{equation}
où $K_{p}$ et $K_{I}$ sont les paramètres de réglages proportionnel et intégral
et les autres symboles correspondent à ce qui est indiqué dans les sections \emph{P controller} et \emph{Integral controller}.

Ces contrôleurs sont 50\% plus lents qu'un contrôleur P seul, mais plus rapides que l'ajout d'un contrôleur intégral \cite{svrcek2014real}.
En effet, si l'on compare son équation avec celle du P controller, le terme $p_0$ a été remplacé par une correction intégrale, ce qui corrige l'erreur automatiquement.

\subsection{MPC controller}
\emph{MPC} est un contrôleur permettant la prédiction des futurs états de différentes variables dans un système.
Celui-ci utilise des stratégies prédictives afin de calculer les prochaines valeurs de sorties possibles.

Contrairement à PID, MPC peut contrôler plusieurs variables, est facile à configurer et peut gérer des changements dans la structure du circuit. % TODO : circuit ?
Par contre, de nombreux modèles MPC ne supportent que les circuits en boucle ouverte.
De plus, ils nécessitent un grand nombre de modèles pour interpoler la réponse et si la commande prédictive est erronée, les performances vont être faibles même si les modèles sont corrects.

Une comparaison entre APM (une variante simple de MPC) et PID a été faite dans \cite{saletovic2014apm}, et montre qu'avec des configurations optimales, APM a apporté un meilleur contrôle que PID.

\section{Détermination des paramètres de PID}

\emph{PID} étant inefficace et donc inutile si ses paramètres (les constantes $K_p$, $K_i$, $K_d$) sont mal fixés, il nous semble important d'aussi étudier les différentes méthodes pour les déterminer. Il existe énormément de méthodes différentes pour configurer \emph{PID}.

Nous étudierons ici trois méthodes manuelles aninsi que deux méthodes heuristiques.

Nous encourageons le lecteur à se réferrer à \cite{shahrokhi2013comparison} qui fait une revue plus détaillée et complète des différentes méthodes pour déterminer les paramètres.


\subsection{La méthode manuelle}

Cette technique s'applique sur un système en fonctionnement.
L'avantage de cette technique est qu'elle ne nécessite aucun calcul mathématique complèxe.

Cependant, elle requiert l'interaction d'un technicien expérimenté, étant donné que chaque variation des paramètres implique de très nombreuses modifications sur le résultat obtenu (croissance initialle, stabilité, délai d'attente jusqu'à la stabilité, dépassement...). \cite{zhong2006pid}


\subsection{Ziegler–Nichols}
% ziegler1942optimum
% silva2007pid
% http://www.chem.mtu.edu/~tbco/cm416/tuning_methods.pdf
Comme la méthode manuelle, Ziegler–Nichols s'applique aussi sur un système en fonctionnement. Cependant, elle utilise des calculs un peu plus complexes que la méthode manuelle mais son avantage est qu'elle nécessite des techniciens moins expérimentés.\cite{ziegler1942optimum} 
En effet, cette méthode requiert peu d'informations sur le système en fonctionnement, étant donné que les formules utilisées découlent d'un nombre exhaustif de simulations de systèmes stables et simples. \cite{silva2007pid}

Néanmoins, très peu d'importance est acordée au bruit lors de la mesure et à la sensibilité, donc il est utile d'évaluer au préalable si cette méthode est adaptée au système étudié. \cite{silva2007pid}

% TODO : Légèrement détailler (peut-être s'inspirer de l'explication)

\subsection{Recherche tabou}


\subsection{Genetic}


\cite{thomas2009position} ont réalisé une comparaison des performances obtenues entre un PID paramétré via la méthode de Ziegler-Nichols et via un algorithme génétique et a montré un temps de réponse beaucoup plus rapide avec cet algorithme.
% TODO : renverser la phrase -> Le temps de réponse de genitic est meilleur \cite{}




\chapter{Méthodes implémentées}
3.un rapport sur les méthodes investiguées et leurs fonctionnement

\section{Contrôle automatique}
\subsection{Bang-bang}

\subsection{PID}


\emph{PID} est de calculer en continu une erreur.
Cette erreur est la différence entre une variable mesurée et un point de consigne arbitraire.
\emph{PID} tente de minimiser l'erreur au long du temps en s'ajustant peu à peu.

Les effets (positifs ou négatifs) de l'ajustement de ces paramètres sont, entre autres \cite{zhong2006pid} :
\begin{enumerate}
\item la stabilité,
\item la précision (l'erreur même quand l'équilibre est atteint),
\item la rapidité (le temps d'attente avant que le système n'arrive à la stabilité)
\end{enumerate}

\subsubsection{Les trois paramètres de PID}
L'utilisation de PID est décrite par la formule suivante\cite{visioli2006practical} :

$$u(t) = K_p e(t) + K_i \int_{0}^{t} e(\tau) d\tau + K_d \frac{de}{dt}$$

Analysons les plusieurs parties de cette formule:

\begin{description}
\item[Action proportionnelle (P) :]
    Cette valeur est proportionnelle à l'erreur de contrôle courante ($K_p e(t)$ ).
\item[Action intégrale (I) :]
    Fait varier la réponse de PID en fonction des valeurs passées de l'erreur de contrôle ($K_i \int_{0}^{t} e(\tau) d\tau$).
\item[Action dérivée (D) :]
    Basée sur les valeurs futures estimées pour l'erreur de contrôle ($K_d \frac{de}{dt}$ ).
\end{description}




\section{Détermination des paramètres de PID}

\subsection{La méthode manuelle}

L'algorithme se déroule en 4 étapes:
\begin{enumerate}
    \item Les valeurs des 3 paramètres sont fixées à $0$.
    \item Le paramètre $K_p$ est incrémenté jusqu'à ce que la sortie du système se mette à osciller.
    On prendra comme valeur pour $K_p$ la moitié de celle obtenue précédemment.
    \item $K_i$ est augmenté jusqu'au moment où l'offset est corrigé dans un temps acceptable pour le système.
    \item Si nécessaire, $K_d$ est augmenté jusqu'au moment où la boucle est suffisamment rapide pour atteindre à nouveau la consigne après une perturbation extérieure.
\end{enumerate}

\subsection{Ziegler–Nichols}


Elle débute comme la méthode manuelle :
\begin{enumerate}
    \item Les valeurs des 3 paramètres sont fixées à $0$.
    \item $K_p$ est augmenté jusqu'à ce que la sortie de la boucle oscille (comme précédemment).
\end{enumerate}

Par contre, cette fois-ci, on va nommer la valeur de $K_p$ ainsi obtenue $K_u$ et la période de l'oscillation $P_u$.
Ensuite, les autres paramètres sont déterminés à l'aide du tableau \ref{tab:ZieglerNicholsTuningFormulas}

\def\tabularxcolumn#1{m{#1}}
\begin{figure}[ht]
    \begin{center}
        \begin{tabularx}{\textwidth}{| c | X | X | X |}
            \hline
            & $K_p$ & $K_i$ & $K_d$\\ \hline
            P & \begin{equation*}\frac{K_u}{2}\end{equation*} & &\\ \hline
            PI & \begin{equation*}\frac{K_u}{2,2}\end{equation*} & \begin{equation*}1,2 \cdot \frac{K_p}{P_u}\end{equation*} &\\ \hline
            PID & \begin{equation*}\frac{K_u}{1,7}\end{equation*} & \begin{equation*}2 \cdot \frac{K_p}{P_u}\end{equation*} & \begin{equation*}K_p \cdot \frac{P_u}{8}\end{equation*} \\
            \hline
        \end{tabularx}
    \end{center}
    \caption{Tableau des formules pour Ziegler–Nichols}
    \label{tab:ZieglerNicholsTuningFormulas}
\end{figure}

\subsection{Recherche Tabou}

Une autre manière utilisée pour déterminer ces paramètres est d'utiliser la recherche tabou. Le but de cette métaheuristique itérative est de trouver des minimums locaux pour une fonction objectif.

Pour se faire, nous appliquons l'algorithme suivant:
\begin{enumerate}
    \item prendre un point au hasard comme solution,
    \item Évaluer les points voisins,
    \item Choisir le meilleur point admissible parmi ces voisins n'étant pas dans la tabou liste,
    \item Si le le meilleur point admissible est meilleur que notre solution, on remplace notre solution par ce nouveau point,
    \item Nous l'ajoutons à notre liste tabou,
    \item Nous retournons en 2 tant que nous ne rencontrons pas notre condition d'arrêt.
\end{enumerate}

Des conditions d'arrêts sont par exemple une limite de temps ou lorsque la différence de score entre deux solutions est inférieure à un certain palier. \cite{glover2007principles}
\cite{bagis2011tabu}


\subsection{Algorithmes génétiques}
Il est également possible de déterminer ces paramètres à l'aide de ce que l'on appelle ``un algorithme génétique''.
Le principe est de générer une population de chromosomes, des êtres virtuels qui possèdent tous 3 ``loci'', représentant chancun l'un des paramètres à déterminer.

Chacun de ces chromosomes sera testé sur l'environnement et recevra un score suivant ses performances.
En fonction de ce score, les chromosomes vont être choisis avec plus ou moins de chance, et procréer.
Cette reproduction pourra éventuellement entraîner un ``enjambement'' entre ces chromosomes, leurs paramètres se mélangeant alors.
Il peut également survenir une ``mutation'': les paramètres vont spontanément changer.

Cette nouvelle population créée va recommencer le cycle précédent et regénérer une autre population, et ainsi de suite un certain nombre de fois.

À la fin de toutes ces itérations, le meilleur dernier chromosome sera sélectionné en fonction de son score.



\chapter{Résultats expérimentaux}
4.les résultats expérimentaux, + un comparaison avec des approches existant.

\section{Méthode expérimentale}

\subsection{Métriques de score}
Pour nous permettre de comparer les différentes méthodes, nous devons introduire des métriques permettant d'attribuer un score à ces méthodes.
Nous utiliserons ici les métriques utilisées par \cite{griffin2003line, mirzal2012pid} :
\begin{description}
    \item[MSE] : La moyenne de l'erreur au carré (Mean of the Squared Error)
    \item[ITAE] : L'intégrale du temps multipliée par la valeur absolue de l'erreur  (Integral of Time multiplied by Absolute Error)
    \item[IAE] : L'intégrale de la valeur absolue de la magnitude de l'erreur (Integral of Absolute Magnitude of the Error)
    \item[ISE] : L'intégrale de l'erreur au carré (Integral of the Squared Error)
    \item[ITSE] : L'intégrale du temps multipliée par l'erreur au carré (Integral of Time multiplied by the Squared Error)
\end{description}

$$
\textbf{MSE} = \frac{1}{T} \int_0^T e(t)^2 dt,
\textbf{ITAE} = \int t \cdot |e(t)| dt,
\textbf{IAE} = \int |e(t)| dt,
\textbf{ISE} = \int e(t)^2 dt,
\textbf{ITSE} = \int t \cdot e(t)^2 dt
$$

\subsection{Dispositif expérimental}

Pour effectuer les mesures


\section{Résultats}
Résulatas pour chaque méthode

pour genetic : comparer en fct du nombre de générations (et le de taille de la pop)
pour tabu, discuter du nombre de tours à faire avant l'arrêt

\chapter{Discussion}
5.l’explication/discussion de ces résultats

\chapter{Conclusion et perspectives}
6.conclusions et perspectives.

\bibliographystyle{apalike}

\bibliography{biblio}
\addcontentsline{toc}{chapter}{Bibliographie}

\end{document}
