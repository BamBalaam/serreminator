\documentclass[a4paper,10pt]{article}
\usepackage[a4paper]{geometry}
\geometry{hscale=0.8,vscale=0.8,centering}

\usepackage[utf8]{inputenc}
\usepackage[T1]{fontenc}
\usepackage[french]{babel}

\usepackage{hyperref}

\title{Modélisation et gestion dynamique d'une serre\\ via un réseau de capteurs.} 
\author{A. Caccia, A. Madeira Cortes, N. Marchant, R. Fontaine} 
\date{ } 

\begin{document}

\maketitle

\vspace{2cm}

\section{Sujet}

Le but du projet est de modéliser et gérer de façon dynamique l'environnement d'une serre. Plusieurs variables seront analysées et contrôlées: humidité du sol et de l'air, luminosité et température. Ces variables seront récupérées via des capteurs installés dans la serre. \\

Lorsque ces variables dépassent des bornes supérieure ou inférieure données, le logiciel construit activera des dispositifs pour les rétablir entre ces bornes. C'est à dire: \\

\begin{enumerate}
	\item Si la température tombe en dessous de la borne donnée, une résistance chauffante est activée.
	\item Si la température dépasse la borne supérieure donnée, une ventilation est activé.
	\item Si l'humidité de l'air est trop basse, un humidificateur est lancé.
	\item Si l'humidité de du sol est trop basse, un système d'irrigation est enclenché.
	\item La luminosité est régulée par une lampe et un système de "stores". (Trouver un meilleur mot!)\\
\end{enumerate}

\section{Implémentation}

MIMO Multiple Input Multiple Output \\

Algorithme PID, variantes, et autres algorithmes.\\

\section{Dispositif de présentation au Printemps des Sciences}

Lors du Printemps des Sciences, on envisage de présenter le projet en montant une serre où grandissent des fraises. \\

L'état des variables contrôlées dans la serre seraient affichées sur un écran, avec un historique visible et les bornes délimitées. \\

<SCREENSHOT DATA HAL HERE> \\

Pour rendre la présentation dynamique et interactive, on pourrait créer des simulations de "perturbation" du système. Celles-ci enclencheraient les dispositifs installés et prouveraient que la serre fonctionne théoriquement en auto-gestion. Des exemples de simulations: \\

\begin{enumerate}
	\item Allumer une lampe pour que le capteur de luminosité active les "stores".
	\item Chauffer la serre avec un dispositif externe (par exemple, un sèche-cheveux) pour que la ventilation s'enclenche. 
	\item Autres... \\
\end{enumerate}

\section{Bibliographie}

\end{document}