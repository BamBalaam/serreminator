\documentclass{beamer}

\usepackage[utf8]{inputenc}
\usepackage[T1]{fontenc}
\usepackage[french]{babel}
\usepackage{pgfplots}
\usepgfplotslibrary{dateplot}
\usepackage{booktabs}
\usepackage[scale=2]{ccicons}
\usepackage{graphicx}

\usetheme{metropolis}

\metroset{inner/titleformat=regular}
\metroset{inner/sectiontitleformat=regular}
\metroset{outer/frametitleformat=regular}

\makeatletter
\renewcommand{\@metropolis@frametitlestrut}{%
    \vphantom{ABCDEFGHIJKLMNOPQRSTUVWXYZabcdefghijklmnopqrstuvwxyz}
  }
\makeatother

\title{Contrôle automatique d'une serre au Printemps des Sciences}
\date{ }
\author{A. Caccia \and A. Madeira Cortes \and N. Marchant \and R. Fontaine}
\institute{Université Libre de Bruxelles}
\begin{document}
	\maketitle

		\begin{frame}{Serreminator - Idée de base}
			\begin{itemize}
        	\item Contrôle en temps réel (environnement d'une serre)
        	\item Modélisation par un système dynamique à variables corrélées
        	\item Contrôlée via des capteurs liés à un Arduino
      		\end{itemize}
		\end{frame}

		\begin{frame}{Serreminator - Présentation au Printemps des Sciences}
			\begin{itemize}
        	\item Une vraie serre (prêtée par Inforsciences)
        	\item Présentation dynamique pour les enfants: perturbation du système
        	\item Graphes en temps réel pour visualiser les variables contrôlées
      		\end{itemize}
		\end{frame}

		\begin{frame}{PID - Présentation de l'algorithme}
			$$u(t) = K_p e(t) + K_i \int_{0}^{t} e(\tau) d\tau + K_d \frac{de}{dt}$$
		\end{frame}

		\begin{frame}{PID - Tuning/Variantes/Prédiction}
			Bla bla
		\end{frame}

        \begin{frame}{Hardware}
            \begin{itemize}
                \item Arduino
                \item Capteurs analogiques
                    \begin{itemize}
                        \item Température
                        \item Luminopsité
                        \item \textit{(Humidité du sol)}
                    \end{itemize}
                \item Capteurs numériques
                    \begin{itemize}
                        \item Humidité de l'air
                        \item Température
                    \end{itemize}
                \item Sorties
                    \begin{itemize}
                        \item LED
                        \item Ventillateur
                        \item Alimentation 5V/12V
                        \item \textit{(Servomoteur)}
                    \end{itemize}
            \end{itemize}
        \end{frame}

		\begin{frame}{Hardware}
			\begin{figure}
                \centering
                \includegraphics[width = 0.9\textwidth]{hardware.pdf}
            \end{figure}
		\end{frame}

		\begin{frame}{HAL}
			\begin{itemize}
				\item Projet du hackerspace toujours en développement
				\item Aide a l'utilisation et au scripting d'un arduino
				\item En deux grosses parties:
				\begin{itemize}
					\item l'arduino
					\item le driver
				\end{itemize}
				\item Facilite la communication entre les deux
			\end{itemize}
		\end{frame}

		\begin{frame}{HAL - Notre contribution}
			\begin{itemize}
				\item Ajout de nouveaux types de senseurs
				\item Aide au développement
				\item Ajouts de nouvelles fonctionnalités
			\end{itemize}
		\end{frame}


\end{document}

